% Options for packages loaded elsewhere
\PassOptionsToPackage{unicode}{hyperref}
\PassOptionsToPackage{hyphens}{url}
%
\documentclass[
  ignorenonframetext,
]{beamer}
\usepackage{pgfpages}
\setbeamertemplate{caption}[numbered]
\setbeamertemplate{caption label separator}{: }
\setbeamercolor{caption name}{fg=normal text.fg}
\beamertemplatenavigationsymbolsempty
% Prevent slide breaks in the middle of a paragraph
\widowpenalties 1 10000
\raggedbottom
\setbeamertemplate{part page}{
  \centering
  \begin{beamercolorbox}[sep=16pt,center]{part title}
    \usebeamerfont{part title}\insertpart\par
  \end{beamercolorbox}
}
\setbeamertemplate{section page}{
  \centering
  \begin{beamercolorbox}[sep=12pt,center]{part title}
    \usebeamerfont{section title}\insertsection\par
  \end{beamercolorbox}
}
\setbeamertemplate{subsection page}{
  \centering
  \begin{beamercolorbox}[sep=8pt,center]{part title}
    \usebeamerfont{subsection title}\insertsubsection\par
  \end{beamercolorbox}
}
\AtBeginPart{
  \frame{\partpage}
}
\AtBeginSection{
  \ifbibliography
  \else
    \frame{\sectionpage}
  \fi
}
\AtBeginSubsection{
  \frame{\subsectionpage}
}
\usepackage{amsmath,amssymb}
\usepackage{iftex}
\ifPDFTeX
  \usepackage[T1]{fontenc}
  \usepackage[utf8]{inputenc}
  \usepackage{textcomp} % provide euro and other symbols
\else % if luatex or xetex
  \usepackage{unicode-math} % this also loads fontspec
  \defaultfontfeatures{Scale=MatchLowercase}
  \defaultfontfeatures[\rmfamily]{Ligatures=TeX,Scale=1}
\fi
\usepackage{lmodern}
\ifPDFTeX\else
  % xetex/luatex font selection
\fi
% Use upquote if available, for straight quotes in verbatim environments
\IfFileExists{upquote.sty}{\usepackage{upquote}}{}
\IfFileExists{microtype.sty}{% use microtype if available
  \usepackage[]{microtype}
  \UseMicrotypeSet[protrusion]{basicmath} % disable protrusion for tt fonts
}{}
\makeatletter
\@ifundefined{KOMAClassName}{% if non-KOMA class
  \IfFileExists{parskip.sty}{%
    \usepackage{parskip}
  }{% else
    \setlength{\parindent}{0pt}
    \setlength{\parskip}{6pt plus 2pt minus 1pt}}
}{% if KOMA class
  \KOMAoptions{parskip=half}}
\makeatother
\usepackage{xcolor}
\newif\ifbibliography
\usepackage{graphicx}
\makeatletter
\def\maxwidth{\ifdim\Gin@nat@width>\linewidth\linewidth\else\Gin@nat@width\fi}
\def\maxheight{\ifdim\Gin@nat@height>\textheight\textheight\else\Gin@nat@height\fi}
\makeatother
% Scale images if necessary, so that they will not overflow the page
% margins by default, and it is still possible to overwrite the defaults
% using explicit options in \includegraphics[width, height, ...]{}
\setkeys{Gin}{width=\maxwidth,height=\maxheight,keepaspectratio}
% Set default figure placement to htbp
\makeatletter
\def\fps@figure{htbp}
\makeatother
\setlength{\emergencystretch}{3em} % prevent overfull lines
\providecommand{\tightlist}{%
  \setlength{\itemsep}{0pt}\setlength{\parskip}{0pt}}
\setcounter{secnumdepth}{-\maxdimen} % remove section numbering
\ifLuaTeX
  \usepackage{selnolig}  % disable illegal ligatures
\fi
\usepackage{bookmark}
\IfFileExists{xurl.sty}{\usepackage{xurl}}{} % add URL line breaks if available
\urlstyle{same}
\hypersetup{
  pdftitle={Story 6: Instacart},
  pdfauthor={Jose Fuentes},
  hidelinks,
  pdfcreator={LaTeX via pandoc}}

\title{Story 6: Instacart}
\author{Jose Fuentes}
\date{2025-04-27}

\begin{document}
\frame{\titlepage}

\begin{frame}{Story 6: Instacart Customer Segmentation}
\phantomsection\label{story-6-instacart-customer-segmentation}
A dataset was given consisting of several files describing customer
purchases which took place at Instacart, an online grocery delivery
service, during a 365 day period prior to 2020. The goal on this
assignment is to perform a customer segmentation analysis to understand
the different types of customer behavior exhibited by Instacart
customers. The dimensionality reduction has to be used,also cluster
analysis, and any other tool that fit to find and visualize customer
segments at Instacart.

The data consists of a partially processed dataset that Instacart posted
to kaggle for a prediction competition. This dataset is being used for a
different purpose.

Note: for this assignment due to pc memory the subsample used is 5000
users.
\end{frame}

\begin{frame}{Datasets description}
\phantomsection\label{datasets-description}
\begin{enumerate}
\tightlist
\item
  user\_features.csv (Pre-processed): Contains user-level features
  derived from the original Instacart data. user\_id: Unique identifier
  for each customer. Food Category Counts (Columns 2-135): Number of
  items ordered by each user across various food categories (Instacart
  ``aisles'') throughout the year. Note: This is a total count and
  doesn't reflect quantities per order. Day of Week Order Counts
  (Columns 136-142): Number of orders placed by each user on each
  specific day of the week.
\item
  Official Instacart Data (Original): aisles.csv: Maps aisle\_id to the
  name of the food category (aisle). departments.csv: Maps
  department\_id to a broader product category (department). Departments
  contain aisles. products.csv: Contains details about each product,
  including its product\_name, aisle\_id, and department\_id.
  orders.csv: Provides high-level information about each order, such as
  user\_id, order\_id, order number for the user, day and hour of the
  order, and days since the previous order. all\_order\_products.csv:
  Contains item-level information for each order, listing all
  product\_ids included in each order\_id and the order in which they
  were added.
\end{enumerate}
\end{frame}

\begin{frame}[fragile]{Preparing data}
\phantomsection\label{preparing-data}
\begin{verbatim}
## Warning: package 'data.table' was built under R version 4.4.2
\end{verbatim}

\begin{verbatim}
## Warning: package 'tidyverse' was built under R version 4.4.3
\end{verbatim}

\begin{verbatim}
## Warning: package 'ggplot2' was built under R version 4.4.3
\end{verbatim}

\begin{verbatim}
## Warning: package 'tidyr' was built under R version 4.4.2
\end{verbatim}

\begin{verbatim}
## Warning: package 'readr' was built under R version 4.4.2
\end{verbatim}

\begin{verbatim}
## Warning: package 'purrr' was built under R version 4.4.3
\end{verbatim}

\begin{verbatim}
## Warning: package 'dplyr' was built under R version 4.4.3
\end{verbatim}

\begin{verbatim}
## Warning: package 'stringr' was built under R version 4.4.2
\end{verbatim}

\begin{verbatim}
## Warning: package 'lubridate' was built under R version 4.4.3
\end{verbatim}

\begin{verbatim}
## -- Attaching core tidyverse packages ------------------------ tidyverse 2.0.0 --
## v dplyr     1.1.4     v readr     2.1.5
## v forcats   1.0.0     v stringr   1.5.1
## v ggplot2   3.5.1     v tibble    3.2.1
## v lubridate 1.9.4     v tidyr     1.3.1
## v purrr     1.0.4     
## -- Conflicts ------------------------------------------ tidyverse_conflicts() --
## x dplyr::between()     masks data.table::between()
## x dplyr::filter()      masks stats::filter()
## x dplyr::first()       masks data.table::first()
## x lubridate::hour()    masks data.table::hour()
## x lubridate::isoweek() masks data.table::isoweek()
## x dplyr::lag()         masks stats::lag()
## x dplyr::last()        masks data.table::last()
## x lubridate::mday()    masks data.table::mday()
## x lubridate::minute()  masks data.table::minute()
## x lubridate::month()   masks data.table::month()
## x lubridate::quarter() masks data.table::quarter()
## x lubridate::second()  masks data.table::second()
## x purrr::transpose()   masks data.table::transpose()
## x lubridate::wday()    masks data.table::wday()
## x lubridate::week()    masks data.table::week()
## x lubridate::yday()    masks data.table::yday()
## x lubridate::year()    masks data.table::year()
## i Use the conflicted package (<http://conflicted.r-lib.org/>) to force all conflicts to become errors
\end{verbatim}

\begin{verbatim}
## Warning: package 'factoextra' was built under R version 4.4.3
\end{verbatim}

\begin{verbatim}
## Welcome! Want to learn more? See two factoextra-related books at https://goo.gl/ve3WBa
\end{verbatim}

\begin{verbatim}
## Warning: package 'cluster' was built under R version 4.4.3
\end{verbatim}

\begin{verbatim}
## Warning: package 'patchwork' was built under R version 4.4.3
\end{verbatim}

\begin{verbatim}
## Warning: package 'ggthemes' was built under R version 4.4.3
\end{verbatim}

\begin{verbatim}
## Warning: package 'pheatmap' was built under R version 4.4.3
\end{verbatim}

\begin{verbatim}
## Warning: package 'tidytext' was built under R version 4.4.2
\end{verbatim}
\end{frame}

\begin{frame}{Determining Number of Clusters - Elbow Method}
\phantomsection\label{determining-number-of-clusters---elbow-method}
\includegraphics{Story6-slides-D608_files/figure-beamer/doc-1.pdf}
\end{frame}

\begin{frame}{KMeans Clustering}
\phantomsection\label{kmeans-clustering}
\includegraphics{Story6-slides-D608_files/figure-beamer/kme-1.pdf}
\end{frame}

\begin{frame}{PCA Visualization with Clusters}
\phantomsection\label{pca-visualization-with-clusters}
\includegraphics{Story6-slides-D608_files/figure-beamer/pcav-1.pdf}
\end{frame}

\begin{frame}{Top products Analysis}
\phantomsection\label{top-products-analysis}
\includegraphics{Story6-slides-D608_files/figure-beamer/tpa-1.pdf}
\end{frame}

\begin{frame}{Clustered Heatmap of Top Products}
\phantomsection\label{clustered-heatmap-of-top-products}
\includegraphics{Story6-slides-D608_files/figure-beamer/chot-1.pdf}
\end{frame}

\begin{frame}{Department Preferences by Cluster}
\phantomsection\label{department-preferences-by-cluster}
\includegraphics{Story6-slides-D608_files/figure-beamer/dch-1.pdf}
\end{frame}

\begin{frame}{Cluster Behavior Profiles Z-scores}
\phantomsection\label{cluster-behavior-profiles-z-scores}
\includegraphics{Story6-slides-D608_files/figure-beamer/cbp-1.pdf}
\end{frame}

\begin{frame}{Enhanced Top Products Heatmap per Cluster}
\phantomsection\label{enhanced-top-products-heatmap-per-cluster}
\includegraphics{Story6-slides-D608_files/figure-beamer/tphch-1.pdf}
\end{frame}

\begin{frame}{Product Sales Patterns by Day of Week}
\phantomsection\label{product-sales-patterns-by-day-of-week}
\includegraphics{Story6-slides-D608_files/figure-beamer/pspdw-1.pdf}
\end{frame}

\begin{frame}{Approximate Revenue Analysis by Cluster}
\phantomsection\label{approximate-revenue-analysis-by-cluster}
\includegraphics{Story6-slides-D608_files/figure-beamer/revenue-1.pdf}
\end{frame}

\begin{frame}{Conclusions}
\phantomsection\label{conclusions}
\begin{enumerate}
[1)]
\tightlist
\item
  4 Distinct Segments: Identified through PCA + KMeans: frequent
  shoppers, organic lovers, weekend buyers, and occasional users.
\item
  Top Products: Organic staples (bananas, strawberries, spinach)
  dominate across clusters.
\item
  Department Patterns: Produce and dairy preferred by most clusters;
  beverages vary by group.
\item
  Time Insights: Clusters differ in shopping hours/days e.g.~Cluster 3
  shops late mornings.
\item
  PCA Effectiveness: First 2 PCs explain \textasciitilde17.7\% variance,
  capturing key behavioral dimensions.
\item
  Actionable Insight: Target promotions by cluster.
\item
  Limitation: Sample size (5k users) may miss niche behaviors.
\item
  The revenue plot indicates Clusters 2 and 4 are big revenue, while
  Clusters 1 and 3 contribute less.
\end{enumerate}
\end{frame}

\end{document}
